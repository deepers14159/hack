\documentclass[10pt, letterpaper, twoside]{article}

\usepackage{fancyhdr}
\usepackage{enumerate}
\usepackage{graphicx}
\usepackage{fullpage}
\usepackage{amsmath}
\usepackage{gensymb}
\usepackage{amssymb}
\usepackage{amsthm}
\usepackage{float}
\usepackage{tikz}
\usepackage{pgf}

\usetikzlibrary{positioning}
\usetikzlibrary{decorations.markings}
\usetikzlibrary{angles,quotes}
\usetikzlibrary{shapes,snakes}


\usepackage{datetime} % \today at \currenttime


\graphicspath{ {Images/} }


\setcounter{section}{0}

\fancyhead{}
\fancyfoot{}

\pagestyle{fancy} % Calling the package

%\fancyhead[CO,CE]{CHM 469 EXAM I}
\fancyfoot[C]{\thepage}
\fancyfoot[RO] {Matthew Carbone}
\fancyfoot[LO] {Theory of Solids HW \#5}
%\fancyhead[C] {REVISED COPY, QUESTION 5: MC CODEUP}

\renewcommand{\headrulewidth}{0pt}
\renewcommand{\footrulewidth}{0pt}

\graphicspath{ {Images/} }

\newcommand{\veca}{\mathbf{a}}
\newcommand{\vecF}{\mathbf{F}}
\newcommand{\vecr}{\mathbf{r}}
\newcommand{\vecx}{\mathbf{x}}
\newcommand{\vecp}{\mathbf{p}}
\newcommand{\vecv}{\mathbf{v}}
\newcommand{\vecq}{\mathbf{q}}
\newcommand{\vecn}{\mathbf{n}}
\newcommand{\vecf}{\mathbf{f}}
\newcommand{\vecy}{\mathbf{y}}
\newcommand{\vecz}{\mathbf{z}}
\newcommand{\vecL}{\mathbf{L}}
\newcommand{\vecA}{\mathbf{A}}
\newcommand{\vecE}{\mathbf{E}}
\newcommand{\vecH}{\mathbf{H}}
\newcommand{\vecR}{\mathbf{R}}
\newcommand{\vecg}{\mathbf{g}}
\newcommand{\vecJ}{\mathbf{J}}
\newcommand{\veck}{\mathbf{k}}
\newcommand{\vecV}{\mathbf{V}}
\newcommand{\vecN}{\mathbf{N}}
\newcommand{\vecT}{\mathbf{T}}
\newcommand{\vectau}{\boldsymbol{\tau}}
\newcommand{\vecomega}{\boldsymbol{\omega}}
\newcommand{\vecmu}{\boldsymbol{\mu}}
\newcommand{\hcal}{\mathcal{H}}
\newcommand{\dif}{\text{d}}
\newcommand{\vecs}{\mathbf{s}}

%% for section styles
\usepackage{titlesec}

\titleformat*{\section}{\large\bfseries}
\titleformat*{\subsection}{\Large\bfseries}
\titleformat*{\subsubsection}{\large\bfseries}
\titleformat*{\paragraph}{\large\bfseries}
\titleformat*{\subparagraph}{\large\bfseries}

\renewcommand{\thesection}{\Roman{section}} 

\begin{document}

%%%%%%%%%%%%%%%%%%%%%%%%%%%%%%%%%%%%%%%%%%%%%%%%%
%%%%%% FORMATS FOR PICTURES, CAPTIONS, TABLES, ETC  %%%%%%%%%%
%%%%%%%%%%%%%%%%%%%%%%%%%%%%%%%%%%%%%%%%%%%%%%%%%

\thispagestyle{empty}
\begin{center}
	 		\noindent\textbf{\Large WHRHS HillsHack 2018 \\ Computational
	 		Science Coding Exercises}\\
			{\large Matthew Carbone}\\
\end{center}

\noindent The following will outline two exercises for you to attempt during
this workshop. \emph{Please do not feel pressured to complete or even necessarily
understand everything going on here}, as this is relatively advanced material,
and the point is that you struggle a bit! That's why we're all here. You should
feel free to discuss any issues or questions you may have with me or your
peers. This work is \emph{not} intended to be completed alone.

Furthermore, please note that this workshop is geared towards computational
science applications, not computer science, so we will not focus so much on
being most optimal with the way we code. We will focus on the bare basics of
the code, and how we can tame it to do what we want. In other words, this is
probably not rigorous from the standpoint of the computer scientist or
software engineer, but it is from the perspective of the physical scientist.
Here are the two exercises.

\section{Introduction to Monte-Carlo}
A critical benchmarking technique in the physical sciences is called
Monte-Carlo. It utilizes the power of random sampling to determine quantities
that are otherwise well defined, but difficult to calculate. We will use a very 
simple example to demonstrate this technique.\\

\textbf{Task:} Estimate the area of a circle, circumscribed inside a square of
side length $s$ by randomly sampling points within the square.\\

What you do know is the following. A circle is defined by a radius $r$, and a 
circle circumscribed inside a square of side length $s$ has a radius
$r = s/2$. If a point $(x, y)$ randomly sampled within the square satisfies
$$ \sqrt{x^2 + y^2} \leq s/2,$$
then that point is contained in the circle. Here are some things you can do:
\begin{enumerate}
	\item Write a Python function that takes the side length $s$ as one 
	argument and $N$, the number of points you wish to sample, as another. It 
	should output your estimation for the area of the circle.
	\item You already know analytically that the area of a circle 
	is $\pi r^2$, so compute the error in your computation for a given $N$.
	Write a function to do this! Then, figure out how to plot the error as a function of $N$. What do you observe?
	\item Think about how you can use symmetry to make this process more 
	efficient.
\end{enumerate}

\section{Horner's Algorithm}
Suppose you have a function $f(x)$, and this function is a very large
(high-degree) polynomial. Maybe it is something like
$$f_n(x) = c_0 + c_1x + c_2 x^2 + ... + c_{n}x^n,$$
where $n$ is a very large number, like 50000, and $c_0, c_1, ..., c_n$ are just
constants that you can choose.

Let's say you want to evaluate this polynomial (just find the value $f_n(x)$
for some $x$). One might imagine that this is not an easy thing
to calculate, let alone even write down on the page, because of the sheer
number of terms. Fortunately, we can write computer code to do it for us, but
how? To really get into the details of how we should do this, we need to
understand the concept of a \emph{floating point operation} (FPO). In this
context, a FPO is an addition, subtraction, multiplication or division of two
objects. For instance, $c_1x$ requires one FPO to execute. A term like
$c_4x^4$, deceptively requires four FPO's. Why? Because the computer doesn't
know any other way to break down the operation $x^4$, except by doing
$ x \times x \times x \times x$. Thus, $c_4x^4$ requires 4 FPO's, and in
general, the $m$th term will require $m$ FPO's to execute.\\

\textbf{Task:} Compute $f_n(x)$ (with the simplification $c_i = 1$ for all $i$) for an arbitrary
$n$ in the shortest amount of computer time possible.\\

This is of course a coding exercise, so I will walk you through to the solution
of our ``shortest time'' possible. First, let's look at how many FPO's the
above method requires. 
\begin{itemize}
	\item If $n=0$, we have no FPO's, since $f = c_0 = 1$.
	\item We have only one FPO if $n=1$, since our constant $c_1 = 1$ and
	therefore there is no operation to evaluate there, but we still have to
	add the two terms together, which counts as one FPO.
	\item What if $n=2$? Three FPO's. Why? Because we get one from multiplying
	$x \times x$ and two from adding the three terms together.
	\item If $n=3$ we have the three FPO's from before \emph{plus} the 
	two from the $n=3$ term, plus one more addition.
\end{itemize}
So in summary, for some $n$, the number of FPO's contributing from
multiplications, additions and all in total are shown:
$$ f_n(x) = 1 + x + x^2 + ... + x^n,$$

\begin{center}
\begin{tabular}{c|l|c|c|c}
$n$ & $f_n(x)$ & FPO$_\times$ & FPO$_+$ & FPO$_\times + $ FPO$_+$\\
\hline
0 & 1 & 0 & 0 & 0 \\
1 & $1 + x$ & 0 & 1 & 1\\
2 & $1 + x + x^2$ & 1 & 2 & 3 \\
3 & $1 + x + x^2 + x^3$ & 3 & 3 & 6\\
4 & $1 + x + x^2 + x^3 + x^4$ & 6 & 4 & 10\\
5 & $1 + x + x^2 + x^3 + x^4 + x^5$ & 10 & 5 & 15\\
6 & $1 + x + x^2 + x^3 + x^4 + x^5 + x^6$ & 15 & 6 & 21\\
\vdots & \vdots & \vdots & \vdots & \vdots\\
$n$ &  $\sum_{m=0}^n x^m$ & $\sum_{m=2}^n m - n + 1$ & $n$ &
 $n(n+1)/2$
\end{tabular}
\end{center}
The above are some rough calculations that I did which show that the amount of
FPO's this algorithm requires as a function of $n$ scales quadratically with
$n$. If we assume (reasonably) that as the number of required FPO's goes up,
so does the amount of time required to run the program, we may say that the 
running time of calculating this polynomial $t$ is proportional to $n^2$:
$$ t \propto n^2.$$

Can we do better? The answer is yes. Horner's algorithm groups polynomial 
terms in the following way; you may confirm for yourself that analytically the
groupings are correct. The groupings severely reduce the scaling, as we can see
by inspecting the following chart.
\begin{center}
\begin{tabular}{c|l|c|c|c}
$n$ & $f_n(x)$ & FPO$_\times$ & FPO$_+$ & FPO$_\times + $ FPO$_+$\\
\hline
0 & 1 & 0 & 0 & 0 \\
1 & $1 + x$ & 0 & 1 & 1\\
2 & $1 + x(1+x)$ & 1 & 2 & 3 \\
3 & $1 + x(1 + x(1 + x))$ & 2 & 3 & 5\\
4 & $1 + x(1 + x(1 + x(1+x)))$ & 3 & 4 & 7\\
5 & $1 + x(1 + x(1 + x(1+x(1+x))))$ & 4 & 5 & 9\\
6 & $1 + x(1 + x(1 + x(1+x(1+x(1+x)))))$ & 5 & 6 & 11\\
\vdots & \vdots & \vdots & \vdots & \vdots\\
$n$ &   & $\max(n-1,0)$ & $n$ &
 $\max(2n-1, 0)$
\end{tabular}
\end{center}
Looking at the total number of FPO's required now, we see that the scaling has
been reduced by a simple algorithmic trick, from $n^2 \rightarrow n$, leading
to
$$ t \propto n.$$
That is the beauty of Horner's algorithm. It is a purely analytical
simplification that drastically reduces the scaling of a process that one might
assume at first glance, is trivially simple.

Here are some things you can do:
\begin{enumerate}
	\item Write Python functions to calculate the polynomial $f_n(x)$ with 
	$c_i=0$ for all $i$, using both the brute force method (the first one
	discussed) and Horner's algorithm. The function should take arguments
	$x$ and $n$.
	\item Compare the time it takes to execute them as a function of $n$.
	\item Write a more general function which takes a list containing
	the arbitrary coefficients $c_i$ (which are now not taken to be 1's), and
	computes $f_n(x)$ (using Horner's algorithm).
\end{enumerate}






\end{document}